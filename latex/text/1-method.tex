\section{Methodology Overview}
\begin{itemize}
    \item taking subreddits to be consist of communities of authors
    \item taking 'interest diversity' to mean the range and frequency of subreddits and author contributes to (comments in)
    \item understand how subreddit communities vary in the extent to authors they engage with a diverse range of subreddits
\end{itemize}

\subsection{How does author diversity vary between subreddits?}
Taking subreddits to consist if comments made by authors, who 'diverse' are subreddits in the number of authors and their relative comments rates. How 'equal' are subreddits in the distribution of comments by authors? 
\subsection{How does interest diversity vary between subreddits?}
For each author I will calculate the number of subreddits they commented in and b) the comment entropy. The subreddit count will give a general idea of author activity on Reddit. The comment entropy will give a better understanding of how equally authors divide their contributions across the subreddits they frequent.


\subsection{How do ‘chamber’ members concentrate within the chamber?}

The literature on echo chamber suggests that confirmation bias encourages people to enter and maintain echo chambers, as we experience pleasure from having our opinions re-affirmed by a group. To test this hypothesis I will determine whether authors who choose to comment in the case study echo chamber spend relatively more time within the chamber than authors in other subreddits. This research question has two sub questions. This is addressed in Section \ref{sub diversity}

\subsubsection{How does author activity vary within subreddits}

The literature suggests that the majority of participation in online communities is heavily unequal. I seek to determine whether this phenomenon hold for echo chambers. I suggest two two possible scenarios: a) chambers are more unequal as they have a more active core of devoted members, or, conversely b) they are more equal as their members are generally equally devoted. I do not yet have a working hypothesis based on the literature on which direction echo chamber participation should be theorised to lean, though my intuition is A.

To measure participation inequality I will first use the same entropy measure from 1.1. This subreddit author entropy will look at the range in number of comments made by all authors. I will then use a more traditional measure of participation inequality, though I have not yet determined which. Options include the gini coefficient and the Pareto principle but I need to do a small literature review to determine which measure has been most appropriately and successfully used for online communities.

\subsubsection{Do chamber members spend more time within the chamber than members of other communities?}

To test this I will measure the in-subreddit ratio for each author, for each subreddit they comment in. The in-subreddit ratio is the number of comments made within the subreddit as a fraction of the total number of comments the author made.




\subsection{Diversity Measures} \label{subreddit vector}
For each subreddit $s_i$ there are $n$ comment authors. For the $nth$ author in subreddit $s_i$ there are $x_n$ number of comments, such that for each subreddit we have the vector:

$$s_i = [x_1, x_2, \cdots x_n]$$

I then use this vector to calculate each of the diversity measures entropy, blau, and gini).


\subsubsection{Entropy(!)} \label{entropy}

"Information entropy is the average rate at which information is produced by a stochastic source of data." The lower the probability of a value, the greater the "information" it carries. The higher the entropy, the greater the disorder or uncertainty.

Within the context of subreddits, if an author only makes a few comments in the subreddit, any new content they bring to subreddit provides relatively more information... Therefore the higher a subreddit's entropy the greater the diversity...


$$Entropy = -\sum p_ilog(p_i)$$

where $p_i$ is the probability of value $x_i$;

$$p_i = \frac{x_i}{\sum x}$$

The maximum entropy for a subreddit is $log(n)$ thus be can normalise entropy as:

$$Entropy_{norm} = \frac{Entropy}{log(n)}$$

\subsubsection{Blau index} \label{blau index}

The Blau index is also known as the Gini-Simpson index, or the Gibbs-Martin index. The Blau index of diversity has a range of 0 to 1, where 0 means the population is perfectly homogeneous, and 1 means the population if perfectly heterogeneous. Therefore the higher the a subreddit's blau, the more heterogeneous the distribution of comments amongst authors, the greater the diversity.


\textcolor{red}{Need to look into using 'true diversity' or normalised blau - may be more informative for more active subreddits (and more correlated to entropy). Can the "effective number of types" show a diversity threshold?}

$$Blau = 1 - \sum (\frac{x_i}{\sum x_n})^{2}$$

\subsubsection{Gini coefficient} \label{gini}

The gini coefficient is a measure of inequality in a distribution. It is a ratio between the values of 0 and 1, where 0 means perfect equality and 1 means perfect inequality. Therefore the higher the value of gini for a given subreddit, the more unequally comments are distributed across authors within the subreddit.

$$Gini = \frac{1}{n}(n+1-2\frac{\sum(n+1-i)y_i}{\sum y_i})$$